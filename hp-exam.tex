\chapter{Examen final}

\section{Le texte suivant a été publié à la fin du chapitre 113~:}


\setlength{\parskip}{.5\baselineskip}

Ceci est votre examen final..

Vous avez 60~heures.

Votre solution doit au moins permettre à Harry d'échapper à la mort immédiate,
alors qu'il est nu, qu'il ne tient que sa baguette et qu'il doit affronter 36 Mangemorts.
plus Lord Voldemort entièrement ressuscité.

\textbf{L'histoire continuera si une solution satisfaisante est proposée d'ici \emph{00:01, heure du Pacifique} (8:01 UTC) le mardi 3 Mars 2015.} 

\textbf{Sinon la fin sera triste et brutale}.

Gardez à l'esprit:
\begin{enumerate}
\item Harry doit réussir grâce à ses propres efforts. Les secours ne viendront pas. Tous ceux qui voudraient aider Harry pensent qu'il assiste au match de Quidditch. 
\item Harry ne peut utiliser les capacités qu'il a déjà démontrées précédemment. Il ne peut pas apprendre la Légitimencie muette en quelques minutes.
\item Voldemort est un méchant et ne peut être persuadé d'agir pour le bien. La fonction d'utilité du seigneur des ténèbres ne peut être changée en lui parlant.
\item Si Harry lève sa baguette ou prononce un mot qui ne serait pas en Fourchelangue, les mangemorts lui tireront dessus immédiatement.
\item Si la ligne temporelle la plus simple est qu'Harry meurt (s'il ne peut attraper son retourneur de temps sans l'utiliser) alors Harry meurt.
\item On ne peut mentir en fourchelangue.
\end{enumerate}
{\setlength{\parindent}{0pt}
\setlength{\parskip}{.5\baselineskip}
En respectant ces contraintes, Harry est autorisé à utiliser toute sa compétence de rationnaliste, en dépit de ses erreurs passées.

Bien sûr la solution "rationelle" est une façon compliquée de parler de la meilleure solution (ou la solution préférable). On préfèrera utiliser une de ces expressions puisque la mot "rationel" se réfère à la façon de raisonner plutôt qu'à une solution particulière. 

D'après le principe de Vinge, si on sait parfaitement ce que fera une personne intelligente, alors on est soi-même au moins aussi intelligent. Demander "que ferait un joueur optimal" ne devrait pas avoir de meilleure réponse que la question "que devrait-on faire?".

C'est à dire, qu'en pratique, "Harry est autorisé à utiliser toute sa compétence de rationaliste" signifie qu'Harry peut résoudre ce problème de la même façon que \emph{vous} le feriez. Si vous pouvez décrire précisement ce qu'il faut faire, alors Harry est autorisé à y penser.

Mais une solution du type "Harry devrait persuader Voldemort de le laisser sortir de la boîte" n'est pas acceptable si vous ne pouvez pas expliquer \emph{comment}.

\textbf{Les règles de Fanfiction.net ne permettent pas de poster plus d'une revue par chapitre. Merci de soumettre \emph{une seule} revue du Chap.~113.}

Afin de garder le plaisir intact, si vous n'avez pas lu de discussion en ligne à propos des chapitres précédents, je vous suggère de \textbf{ne pas} le faire mais de répondre à cet examen seul, ne pas lire les autres revues et d'attendre le Ch.~114 afin d'évaluer votre réponse.

Je vous souhaite bonne chance, ou plutôt: bonne réflexion.

Ch.~114 sera publié à \textbf{10h heure du Pacifique (6h UTC) le mardi 3 mars 2015.}

Addendum:

Si vous êtes en pleins examens: même en considérant l'effet du témoin, je m'attends à ce que "la non-participation de tous ceux ayant des choses plus urgentes à régler" n'impacte que marginalement la solution. Ceci car l'effet marginal de cerveaux supplémentaires dans une population assez large de participants est faible.

\textbf{Donc, si vous n'avez pas de temps à y consacrer, ne le faîtes pas.} Comme tous les écrivains, j'apprécie tout particulièrement faire suer mes lecteurs mais je ne voudrais pas que cela vous porte préjudice.

De même, si vous n'appréciez pas participer à ce genre de jeu, abstenez-vous! Certains \emph{aiment} cela. Revenez simplement dans quelques jours. Cela devrait être évident bien sûr.

Je vous rappelle que vous avez plusieurs heures de réflexion. Utilise la méthode "Se retenir de proposer des solutions" Luke.

Et j'insiste vraiment, Harry ne peut pas inventer de nouveaux pouvoirs magiques, ou dépasser les limites citées précédement, en quelques minutes.
\later
Sans surprises, trop de propositions ne respectaient pas cette contrainte.

Pour lire le résultat: \href{http://www.reddit.com/r/HPMOR/comments/2xnyi0/113_help_my_evil_plan_has_worked_all_too_well/}{sur le sous-reddit /r/HPMOR}. 
Si vous lisez ceci sur un support où le texte précédent n'est pas un lien, vous pouvez ouvrir \url{http://www.reddit.com/r/HPMOR} and rechercher “Help! My evil plan has worked all too well!”
}
%  LocalWords:  Ch
