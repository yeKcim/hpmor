\chapter{Réductionnisme}

\iftoggle{embedepigraphs}{Tout ce qui peut Rowlinguer, Rowlinguera.

\iftoggle{embedepigraphs}{\newpage}{}
}{}

\lettrine[ante=«~]{D}{'accord}, dit Harry en déglutissant. D'accord Hermione, ça suffit, tu peux arrêter.~»

La pile de sucre blanc placée devant Hermione n'avait changé ni de forme ni de couleur, et pourtant elle s'était concentrée avec plus de force que Harry ne l'avait jamais vue le faire, ses paupières serrées l'une contre l'autre, des perles de sueur sur le front, la main tremblante, agrippée sur la baguette…

«~Hermione, \emph{arrête}~! Ça ne va pas marcher, je ne pense pas qu'on puisse créer des choses qui n'existent pas encore~!~»

La main de Hermione relâcha lentement sa prise sur la baguette.

«~J'ai eu l'impression de le sentir, dit-elle d'une voix qui était à peine un murmure. J'ai eu l'impression que je l'ai senti commencer à se métamorphoser, pendant juste une seconde.~»

Il y avait une boule coincée dans la gorge de Harry.

«~Tu l'as probablement imaginé. Tu espérais trop fort.

--- Probablement~», dit-elle. Elle avait l'air d'avoir envie de pleurer.

Harry se saisit lentement de son porte-mine, puis il se pencha au-dessus de la feuille de papier sur laquelle on pouvait voir une liste d'éléments presque tous biffés et il traça une ligne en travers de la ligne qui disait «~REMÈDE CONTRE LA MALADIE D'ALZHEIMER~».

Ils n'auraient pas pu donner une pilule métamorphosée à quelqu'un. Mais la métamorphose, du moins celle qu'ils étaient capables de faire, n'enchantait pas les cibles -- elle n'aurait pas métamorphosé un balai normal en un balai volant. Donc si Hermione était parvenue à fabriquer la pilule, cela aurait été une pilule \emph{non-magique}, une pilule qui aurait fonctionné pour des raisons ordinaires et physiques. Ils auraient pu fabriquer des pilules en secret pour le compte d'un laboratoire scientifique Moldu, les laisser \emph{étudier} les pilules et essayer d'effectuer une rétro-ingénierie sur celles-ci avant que la métamorphose ne se dissipe… pas un habitant des mondes magique et Moldu n'auraient à savoir qu'il y avait eu de la magie là-dessous, ce ne serait qu'une percée scientifique de plus…

Ça n'aurait de toute façon pas été le genre de chose à laquelle un sorcier aurait pensé. Ils ne respectaient pas particulièrement les \emph{simples assemblages d'atomes}, ils ne voyaient pas les objets \emph{physiques} non-enchantés comme des objets de pouvoir. Si ça n'était pas magique, ça n'était pas intéressant.

Plus tôt, Harry avait, dans le plus grand secret -- il ne l'avait même pas dit à Hermione -- essayé de métamorphoser de la nanotechnologie à la Éric Drexler (il avait essayé de construire une nano-usine de bureau, évidemment, pas de petits assembleurs auto-réplicateurs, il n'était pas dingue). Si ça avait marché, ça aurait été la divinité du premier coup.

«~C'est tout pour aujourd'hui, hein~?~» dit Hermione. Elle était affalée sur sa chaise, tête appuyée contre le dossier~; et son visage exprimait sa fatigue, ce qui était très inhabituel chez Hermione. Elle aimait à prétendre qu'elle n'avait pas de limites, ou du moins elle aimait le faire quand Harry était dans les parages.

«~Un dernier, dit prudemment Harry, mais celui-ci est petit, et puis ça pourrait vraiment marcher. Je l'ai gardé pour la fin parce j'espérais qu'on puisse finir sur une note positive. C'est quelque chose de réel, pas comme des sabres lasers. Ils en ont déjà fabriqué en laboratoire, au contraire du remède contre Alzheimer. Et c'est une substance générique, pas quelque chose de spécifique comme les livres perdus dont tu as essayé de métamorphoser des exemplaires. J'ai fait un diagramme de la structure moléculaire pour que tu la visualises. On veut juste en faire un \emph{plus long} que celui qui a déjà été fait, avec tous les tubes alignés et les extrémités imbriquées dans du diamant.~» Harry exhiba une feuille de papier millimétré.

Hermione se redressa sur sa chaise et étudia la feuille en fronçant les sourcils. «~Ce sont \emph{tous} des atomes de carbone~? Et Harry, comment ça s'appelle~? Je ne peux pas en métamorphoser si je ne connais pas son nom.~»

Harry eut une moue dégoûtée. Il avait encore du mal à se faire à ce genre de chose, ça n'aurait pas dû avoir d'importance de connaître le \emph{nom} de quelque chose si on savait ce que \emph{c'était}. «~On les appelle des nanotubes de carbone. C'est un genre de fullerène qui a été découvert cette année. C'est environ cent fois plus résistant que l'acier et ça pèse un sixième de son poids.~»

Hermione releva les yeux de la feuille de papier millimétré.

«~C'est \emph{réel}~?

--- Ouais, dit Harry, juste difficile à faire pour des Moldus. Si on pouvait obtenir une quantité suffisante de ce matériau, on pourrait l'utiliser pour construire un ascenseur spatial jusqu'à une altitude d'orbite géostationnaire, voir encore plus haut puisqu'en termes de delta-v c'est déjà à mi-chemin entre le sol et n'importe où dans le système solaire. En plus, on pourrait propulser des satellites fournisseurs d'énergie solaire comme si c'étaient des confettis.~»

Hermione fronçait à nouveau les sourcils.

«~C'est \emph{sûr}, ce truc~?

--- Je ne vois pas pourquoi ça ne le serait pas, dit Harry. Un nanotube de carbone est juste une feuille de graphite roulée en un tube circulaire, pour faire simple, et le graphite est la même matière que celle utilisée pour faire des crayons…

--- Je \emph{sais} ce qu'est le graphite, Harry~», dit Hermione. Elle se recoiffa d'un air distrait, ses sourcils toujours froncés tandis qu'elle fixait la feuille de papier.

Harry glissa sa main dans une poche de sa robe et fit apparaître un fil blanc dont chaque extrémité était reliée à un petit anneau de plastique. Il avait ajouté des gouttes de superglue à chaque point de contact entre le fil et les anneaux pour en faire un objet unique qui pourrait être métamorphosé comme un tout. Si Harry se souvenait bien, le cyanoacrylate fonctionnait en formant des liaisons covalentes, et c'était le plus près qu'on pourrait jamais approcher de la notion “d'objet solide” dans un monde en définitive composé de petits atomes individuels.

«~Quand tu es prête, dit Harry, essaie de métamorphoser ça en un ensemble de nanotubes de carbone alignés et incrusté dans deux anneaux de diamant solide.

--- Très bien… dit lentement Hermione. Harry, j'ai l'impression que je viens de laisser passer quelque chose.~»

Harry haussa les épaules d'un air impuissant. \emph{Peut-être que tu es simplement fatiguée}. Mais il avait assez de bon sens pour ne pas le dire à voix haute.

Hermione posa sa baguette contre un des anneaux de plastique et le regarda pendant un moment.

Deux petits cercles de diamant scintillant étaient posés sur la table, reliés par un long fil noir.

«~Ça a changé~», dit Hermione. Elle avait l'air d'essayer d'être enthousiaste mais d'être en fait à court d'énergie. «~Et maintenant~?~»

Harry se sentit un peu découragé par le manque de passion de sa partenaire, mais il fit de son mieux pour ne pas le montrer~; peut-être que le même processus marcherait à l'inverse pour lui remonter le moral. «~Maintenant, je le teste pour voir s'il supporte des poids.~»

Il y avait un tréteau que Harry avait installé plus tôt afin de faire une expérience avec des bâtons de diamant -- on pouvait facilement créer des objets de diamant solide en utilisant la métamorphose, c'est juste qu'ils ne duraient pas. L'expérience précédente avait mesuré si la métamorphose d'un long bâton de diamant en un même bâton plus court permettait de soulever un poids lourd pendant que le bâton se contractait, c'est-à-dire si on pouvait métamorphoser un objet en tension, ce qui s'était révélé être en effet possible.

Harry enroula précautionneusement un cercle de diamant scintillant autour du large crochet en métal qui se trouvait en haut de l'installation puis il attacha un épais cintre de métal à l'anneau du bas et il commença à attacher des poids au cintre.

(Harry avait demandé aux jumeaux Weasley de lui métamorphoser cet appareillage, et les jumeaux lui avaient jeté un regard incrédule comme s'ils n'arrivaient pas à imaginer pour quelle sorte de farce il \emph{pouvait bien} avoir besoin de ça, mais ils n'avaient posé aucune question. Et selon eux, leurs métamorphoses duraient environ trois heures, donc Harry et Hermione avaient encore un bon moment devant eux).

«~Cent kilos, dit Harry une minute plus tard. Je ne pense pas qu'un fil d'acier aussi fin que celui-ci aurait pu en soutenir autant. On devrait pouvoir monter encore plus haut, mais c'est tout le poids dont je dispose.~»

Il y eut plus de silence.

Harry se redressa, revint à leur table, s'assit dans sa chaise et cocha cérémonieusement la ligne qui comportait les mots “nanotubes de carbone”.

«~Voilà, dit Harry. \emph{Celui-là} a fonctionné.

--- Mais ce n'est pas vraiment \emph{utile}, n'est-ce pas~?~» dit Hermione depuis sa chaise, tête reposée sur ses mains. «~Je veux dire que même si on le donnait à un scientifique, il ne pourrait pas apprendre comment faire plein de nanotubes en étudiant les nôtres.

--- Il pourrait apprendre \emph{quelque chose}, dit Harry. Hermione, \emph{regarde ça}, ce petit fil qui soutient tout ce poids, on vient de faire quelque chose qu'aucun laboratoire Moldu ne pourra jamais faire…

--- Mais n'importe quelle sorcière le pourrait~», dit Hermione. Son épuisement filtrait maintenant à travers sa voix. «~Harry, je ne pense pas que ça fonctionne.

--- Tu veux dire notre relation~? dit Harry. Génial~! Cassons alors.~»

Cela lui arracha un léger sourire.

«~Je veux dire nos recherches.

--- Oh, Hermione, comment \emph{peux-tu}~?

--- Tu es mignon quand tu es méchant, dit-elle. Mais Harry, c'est de la folie, j'ai douze ans, tu en as onze, c'est \emph{idiot} de croire qu'on va découvrir quelque chose dont personne ne s'est déjà rendu compte.

--- Es-tu vraiment en train de dire qu'on devrait abandonner notre démêlage des secrets de la magie après avoir essayé pendant moins d'un \emph{mois}~?~» dit Harry en essayant de mettre une note de défi dans sa voix. Pour être honnête, il sentait une partie de la fatigue de Hermione. Aucune des \emph{bonnes} idées ne fonctionnait jamais. Il avait fait une seule découverte qui mérite d'être mentionnée, le motif Mendélien, et il ne pouvait pas en parler à Hermione sans briser sa promesse envers Drago.

«~Non~», dit Hermione. Son jeune visage semblait très sérieux et très adulte. «~Je veux dire que pour l'instant on devrait \emph{étudier} toute la magie que les sorciers connaissent déjà pour pouvoir faire ce genre d'expériences après que l'on aura terminé nos études à Poudlard.

--- Euh… dit Harry. Hermione, je déteste avoir à le formuler ainsi, mais imagine qu'on ait décidé de s'abstenir de faire des recherches et d'attendre plus tard, et que la première chose qu'on essayait après avoir terminé nos études, c'était de métamorphoser un remède contre la maladie d'Alzheimer, et que ça \emph{marchait}. On se sentirait… je ne pense pas que le mot \emph{stupide} décrive de façon adéquate ce qu'on ressentirait. Et s'il y a autre chose de semblable, et que ça fonctionne~?

--- Ce n'est pas \emph{juste}~!~» dit Hermione. Sa voix tremblait comme si elle était sur le point d'éclater en sanglots. «~Tu ne \emph{peux pas} imposer ça à des gens~! Ce n'est pas notre \emph{travail} de faire ce genre de chose, on est des \emph{enfants}~!~»

Pendant un instant, Harry se demanda ce qui se passerait si quelqu'un disait à Hermione qu'elle devait combattre un Seigneur des Ténèbres immortel, si elle deviendrait un de ces héros pleurnicheurs apitoyés sur leur sort que Harry ne supportait pas de voir apparaître dans ses livres.

«~Bref~», dit Hermione. Sa voix trembla. «~Je ne veux pas continuer à faire ça. Je ne crois pas que les enfants peuvent faire des choses que les adultes ne peuvent pas faire, ça n'existe que dans les histoires.~»

Il y eut un silence dans la salle de classe.

Hermione commença à avoir l'air un peu effrayée, et Harry sut que son visage venait d'adopter une expression plus froide.

Ça n'aurait pas fait aussi mal si la même pensée n'était pas déjà venue à Harry -- que, bien que trente ans soit un peu vieux pour un révolutionnaire scientifique et vingt ans à peu près dans la norme, bien qu'il y ait des gens qui obtenaient leur doctorat à dix-sept ans et que des héritiers de quatorze ans avaient été d'excellents rois et d'excellents généraux, il n'y avait pas vraiment qui que ce soit qui soit entré dans les livres d'histoires à l'âge de onze ans.

«~Très bien, dit Harry. Trouver comment faire quelque chose qu'un adulte ne peut pas faire. C'est ça ton défi~?

--- Ce n'est pas ce que je voulais dire~», dit Hermione, sa voix s'échappant sous la forme d'un chuchotement apeuré.

Harry fit un immense effort et arracha son regard à celui de Hermione. «~Je ne suis pas en colère contre \emph{toi}~», dit-il. Sa voix était froide en dépit de ses efforts. «~Je suis en colère contre, je ne sais pas, contre tout. Mais je ne suis pas prêt à perdre. Perdre n'est pas toujours la meilleure des stratégies. Je trouverai comment faire quelque chose qu'un sorcier adulte ne peut pas faire, et je reviendrai te voir. Qu'est-ce que tu en penses~?~»

Il y eut encore plus de silence.

«~D'accord~», dit Hermione, sa voix un peu vacillante. Elle s'extirpa de sa chaise et alla jusqu'à la porte de la salle de classe abandonnée où ils avaient travaillé. Sa main atteint la poignée. «~On est toujours amis, oui~? Et si tu ne trouves rien…~»

Sa voix s'arrêta.

«~Alors nous étudierons ensemble~», dit Harry. Sa voix était maintenant encore plus froide.

«~Euh, au revoir pour l'instant, alors~», dit Hermione, et elle sortit rapidement de la pièce et referma la porte derrière elle.

Parfois, Harry détestait avoir un côté obscur, même quand il était à l'intérieur de celui-ci.

Et la partie de lui qui avait pensé exactement à la même chose que Hermione, que non, les enfants \emph{ne pouvaient pas} faire ce dont les adultes n'étaient pas capables, cette partie disait toutes les choses que Hermione n'avait pas dites parce qu'elle était trop effrayée, comme \emph{C'est un défi sacrément difficile que tu viens de te faire à toi-même} et \emph{mon gars tu vas vraiment avoir l'air fin} et \emph{au moins comme ça tu sauras que tu as échoué}.

Et la partie de lui qui n'aimait pas perdre répondit d'une voix très froide, \emph{Très bien, tu peux te taire et admirer}.

\later

C'était presque l'heure du déjeuner et Harry s'en fichait. Il n'avait même pas pris la peine de prendre une barre énergétique dans sa bourse. Son estomac pouvait bien supporter d'être un peu affamé.

Le monde des sorciers était petit, ils ne pensaient pas comme des scientifiques, ils ne connaissaient pas la science, ils ne remettaient pas en question cet autour de quoi ils avaient grandi, ils n'avaient pas mis de coques protectrices autour de leur machines à remonter le temps, ils jouaient au Quidditch, toute l'Angleterre magique était plus petite qu'une petite ville Moldue, la plus grande école de magie n'éduquait que jusqu'à l'âge de dix-sept ans, ce n'était pas \emph{idiot} de remettre ça en question à l'âge de onze, ce qui était \emph{idiot}, c'était de \emph{partir du principe} que les sorciers savaient ce qu'ils faisaient et qu'ils avaient épuisé toutes les solutions faciles qu'un polymathe aurait su voir.

L'étape une avait été de faire une liste de toutes les contraintes magiques dont Harry pouvait se souvenir, toutes les choses qu'on n'était pas censé pouvoir faire.

Étape deux, noter les contraintes qui semblaient être les \emph{moins} logiques d'un point de vue scientifique.

Étape trois, ordonner les contraintes en fonction de la probabilité qu'un sorcier aurait de les remettre en question s'il ne connaissait \emph{pas} la science.

Étape quatre, trouver des moyens de s'attaquer à ces contraintes.

\later

Hermione était encore un peu tremblante lorsqu'elle s'assit à côté de Mandy à la table Serdaigle. Son déjeuner avait consisté de deux fruits (des tranches de tomates et des mandarines épluchées), trois légumes (des carottes, des carottes et encore des carottes), un morceau de viande (des brochettes de Dirico dont elle avait précautionneusement ôté la panure fort mauvaise pour la santé), et un petit morceau de gâteau au chocolat qu'elle avait mérité en mangeant les autres aliments.

Ça n'avait pas été aussi terrible que le cours de Potions --- elle avait encore parfois des \emph{cauchemars} au sujet de ce cours. Mais cette fois-ci, c'est \emph{elle} qui l'avait provoqué, et \emph{elle s'était sentie devenir sa cible}. Juste un instant, avant que l'horrible ténèbre glacée ne détourne le regard et ne lui dise qu'elle n'était pas en colère contre elle, parce qu'elle n'avait pas voulu lui faire peur.

Et elle avait toujours l'impression qu'elle avait raté quelque chose, quelque chose de très important.

Mais ils n'avaient enfreint aucune des règles de la Métamorphose… non~? Ils n'avaient fait aucun liquide, aucun gaz, ils n'avaient reçu aucun ordre venant du professeur de Défense…

La \emph{pilule}~! C'était quelque chose de mangeable~!

… enfin non, personne n'aurait mangé une pilule qui traînait, et ça n'avait pas vraiment \emph{marché}, ils auraient pu se contenter de jeter \emph{Finite Incantatem} s'ils avaient réussi, mais elle devrait quand même en faire part à Harry et s'assurer de ne jamais en faire mention au professeur McGonagall, au cas où ils ne seraient alors plus jamais autorisés à étudier la Métamorphose…

Hermione commençait à se sentir vraiment malade. Elle repoussa son assiette, elle ne pouvait pas déjeuner dans cet état.

Et elle ferma ses yeux et commença à réciter mentalement les règles de la Métamorphose.

«~\emph{Je ne métamorphoserai jamais rien en un liquide ou en un gaz}.

--- \emph{Je ne métamorphoserai jamais rien qui ressemble à de la nourriture ou quoi que ce soit qui soit destiné à entrer dans un corps humain.}~»

Non, ils n'auraient \emph{vraiment pas} dû essayer de métamorphoser la pilule, ou au moins ils auraient dû se \emph{rendre compte}… elle avait été tellement saisie par l'idée brillante de Harry qu'elle n'avait pas \emph{pensé}…

La sensation désagréable dans l'estomac de Hermione empirait. Il y avait une intuition dans son esprit, comme quelque chose qui aurait flotté à la limite de sa conscience, une perception sur le point de s'inverser, une jeune femme sur le point de devenir une vieillarde, un vase sur le point de devenir deux visages…

Et elle continua de se remémorer les règles de la Métamorphose.

\later

Les jointures des doigts de Harry étaient déjà blanches quand il arrêta d'essayer de métamorphoser l'air en face de lui en un trombone. Ça n'aurait pas été prudent de métamorphoser le trombone en gaz bien sûr, mais Harry ne voyait aucune raison pour laquelle ce ne serait pas prudent d'effectuer l'opération inverse. C'était juste considéré comme étant \emph{impossible}. Mais pourquoi pas~? L'air était une substance aussi réelle qu'une autre…

Bon, peut-être que cette limitation \emph{avait} un sens. L'air était désorganisé, toutes les molécules changeaient constamment leurs relations les unes aux autres. Peut-être qu'il était impossible d'imposer une nouvelle forme à une substance à moins que la substance ne reste stable assez longtemps pour être maîtrisée, même si les atomes des solides vibraient eux aussi constamment…

Plus Harry échouait, plus il se sentait devenir froid, plus les choses semblaient devenir claires.

Très bien. Suivant sur la liste.

On pouvait uniquement transformer les objets dans leur intégralité. On ne pouvait pas transformer la \emph{moitié} d'une allumette en épingle, il fallait métamorphoser \emph{l'objet entier}. Quand Harry s'était retrouvé enfermé dans cette salle par Drago, ça avait été la raison pour laquelle il n'avait pas pu juste métamorphoser en éponge une coupe cylindrique des murs et percer un trou dans la pierre assez large pour passer au travers de celui-ci. Il aurait fallu imposer une nouvelle forme au mur dans son ensemble et peut-être à toute une section de Poudlard juste pour changer ce petit bout.

Et c'était \emph{ridicule}.

\emph{Les choses étaient composées d'atomes}. Plein de petits points. Il n'\emph{existait} pas de continuité, il n'\emph{existait} pas de solidité, juste des forces électromagnétiques tenant des petits points reliés les uns aux autres…

\later

Mandy Brocklehurst interrompit le mouvement de sa fourchette en chemin vers sa bouche. «~Euh~», dit-elle à Su Li, qui était assise de l'autre côté de l'espace maintenant vide situé à sa droite, «~qu'est-ce qui lui arrive, à Hermione~?~»

\later

Harry voulait tuer sa gomme.

Il avait essayé de métamorphoser en acier un seul petit endroit du rectangle rose, séparé du reste du caoutchouc, et la gomme ne voulait pas coopérer.

Ça ne \emph{pouvait pas} être une véritable limitation, il \emph{fallait} qu'elle soit seulement conceptuelle.

\emph{Les choses étaient composées d'atomes}, et chaque atome était une petite partie distincte. Les atomes étaient reliés entre eux par une brume quantique chargée d'électrons pour les liaisons covalentes, ou parfois juste par du magnétisme à courte portée pour les liaisons ioniques, ou par les forces de van der Waals.

Si on voulait en venir là, les protons et les neutrons à l'intérieur du nucléus étaient de petites choses distinctes. Les quarks à l'intérieur des protons et des neutrons étaient de petites choses distinctes~! Il n'y avait tout simplement \emph{rien} dans la réalité, dans le monde-tel-qu'il-était, qui correspondait au concept humain d'objet solide. Tout n'était que petits points.

Et la Métamorphose libre était en premier lieu une opération de l'esprit, non~? Pas de mots, pas de gestes. Seul le pur concept de forme, maintenu fermement séparé de celui de substance, imposé à une substance visualisée sans sa forme. Ça et la baguette et ce quelque chose qui faisait de vous un sorcier.

Si les sorciers ne pouvaient pas métamorphoser des parties d'ensembles, s'ils pouvaient uniquement transformer ce que leurs esprits percevaient comme étant des objets entiers, c'était parce qu'ils ne savaient pas, \emph{du fond de leur âme}, qu'au fond, tout n'était qu'atomes.

Et Harry n'avait toujours pas réussi à transformer ce petit bout de la gomme, la métamorphose n'allait nulle part.

\emph{C'était. Ridicule.}

Les jointures des doigts de Harry blanchissaient. Les résultats expérimentaux \emph{vides de sens} commençaient à le rendre \emph{malade}.

Peut-être que le fait qu'une \emph{partie} de son esprit pensait encore en termes d'objets entiers empêchait la métamorphose d'avoir lieu. Il avait pensé à une collection d'atomes qui était une \emph{gomme}. Il avait pensé à une collection qui était \emph{un petit bout}.

L'heure était venue de faire monter la pression d'un cran.

Harry appuya sa baguette contre le petit bout de la gomme et il essaya de voir à travers l'illusion que les non-scientifiques considéraient être la réalité, le monde des bureaux et des chaises, de l'air, des gommes et des gens.

Quand vous vous promeniez dans un parc, le monde immersif qui vous entourait était quelque chose qui existait uniquement dans votre esprit sous la forme d'une structure de neurones en train de s'allumer. La sensation d'un ciel bleu clair n'était pas quelque chose qui se trouvait haut au-dessus de vous, c'était quelque chose situé dans votre cortex visuel, et votre cortex visuel était à l'arrière de votre cerveau. Toutes les sensations de ce monde radieux avaient en réalité lieu dans la grotte d'os silencieuse que vous appeliez votre crâne, le lieu où \emph{vous} viviez et que vous ne quittiez \emph{jamais}. Si vous vouliez vraiment dire bonjour à quelqu'un, à la \emph{vraie personne}, vous ne leur serreriez pas la main, vous frapperiez gentiment contre leur crâne et vous diriez~: «~Comment ça va là-dedans~?~». C'était ça, les gens, et c'était là qu'ils vivaient vraiment. Et \emph{l'image} du parc au travers duquel vous pensiez \emph{vous promener} était quelque chose que vous visualisiez à l'intérieur de votre cerveau tandis que celui-ci traitait les signaux envoyés depuis vos yeux et votre rétine.

Ce n'était pas un \emph{mensonge}, comme le pensaient les Bouddhistes, il n'y avait pas quelque chose de mystique et terrible, quelque chose d'inattendu caché derrière le voile de Maya, ce qui se cachait derrière l'illusion du parc, c'était juste le \emph{vrai} parc, mais vous voyiez quand même une \emph{illusion}.

Harry n'était pas assis dans la salle.

Il ne regardait pas la gomme.

Harry était dans le crâne de Harry.

Il ressentait l'image traitée que son cerveau avait décodée à partir des signaux envoyés par sa rétine.

La vraie gomme était ailleurs, ailleurs que là où se trouvait son image.

Et la vraie gomme n'était pas semblable à l'image qu'en avait le cerveau de Harry. L'idée d'une gomme \emph{solide} était quelque chose qui existait uniquement dans son propre cerveau, dans le lobe pariétal, qui gérait son sens des formes et des espaces. La vraie gomme était une collection d'atomes reliés par des forces électromagnétiques et partageant des liaisons covalentes, pendant que non loin, des molécules d'air rebondissaient les unes contre les autres et rebondissaient contre les molécules-de-gomme.

La véritable gomme était très loin, et Harry, depuis son crâne, ne pourrait jamais tout à fait l'atteindre, il pouvait seulement imaginer des images la concernant. Mais \emph{sa baguette avait le pouvoir}, elle pouvait modifier les choses \emph{dans le monde}, c'étaient seulement les \emph{concepts} de Harry qui la \emph{limitaient}. Quelque part, derrière le voile de Maya, la \emph{vérité} cachée derrière le concept que Harry nommait “ma baguette” touchait la collection d'atomes que l'esprit de Harry nommait “un bout de la gomme”, et si cette baguette pouvait transformer la collection d'atomes que Harry considérait être “la gomme dans son ensemble”, il n'y avait aucune raison pour laquelle la baguette ne pourrait pas aussi changer l'autre collection…

La métamorphose ne fonctionnait toujours pas.

Les dents de Harry se serrèrent, et il fit monter la pression d'un \emph{autre} cran.

Le concept que Harry avait de la gomme en tant qu'un objet unique était \emph{absurdité évidente}.

C'était une carte qui ne correspondait pas, ne \emph{pouvait pas} correspondre au territoire.

Les êtres humains se représentaient le monde en utilisant des niveaux stratifiés d'organisation, ils avaient des \emph{pensées} distinctes concernant la façon dont les pays, les gens, les organes, les cellules, les molécules et les quarks fonctionnaient.

Lorsque le cerveau de Harry avait besoin de penser à la gomme, il pensait à toutes les règles qui gouvernaient les gommes, comme “les gommes peuvent effacer les marques de crayon”. Ce n'était que si le cerveau de Harry avait besoin de prédire ce qui se passerait à un niveau chimique plus bas qu'il commencerait à penser -- comme si c'était un fait distinct du précédent -- à des molécules de caoutchouc.

Mais tout cela était dans son \emph{esprit}.

L'esprit de Harry avait peut-être des \emph{croyances} séparées concernant les règles qui gouvernaient les gommes, mais il n'y avait pas de \emph{loi physique} \emph{séparée} qui gouvernait les gommes.

L'esprit de Harry représentait la réalité en utilisant plusieurs niveaux d'organisation, avec des croyances différentes pour chaque niveau. Mais tout cela était dans la \emph{carte}, le véritable territoire ne ressemblait pas à ça, \emph{la réalité elle-même} avait \emph{un seul} niveau d'organisation~: les quarks. C'était un processus de bas niveau unifié qui obéissait à des règles mathématiques simples.

Ou du moins c'était ce que Harry avait cru avant de découvrir l'existence de la magie, mais la gomme n'était pas magique.

Et même si la gomme \emph{avait} été magique, l'idée qu'il puisse \emph{vraiment} exister une gomme solide était \emph{impossible}. Les choses comme les gommes ne \emph{pouvaient pas} être des éléments de base de la réalité, elles étaient trop grosses et trop compliquées pour être des atomes, il \emph{fallait} qu'elles soient composées de sous-parties. Des choses \emph{fondamentalement complexes} ne pouvaient pas exister. La croyance implicite qu'avait le cerveau de Harry, celle qui disait que la gomme était un objet unique, était tout simplement \emph{fausse}, c'était une confusion carte-territoire, la gomme n'existait que sous la forme d'un concept distinct dans la \emph{représentation} à plusieurs niveaux que Harry avait du monde, pas sous la forme d'un élément de la réalité à un seul niveau.

… la métamorphose \emph{ne fonctionnait toujours pas}.

Harry respirait lourdement, une métamorphose échouée était presque aussi épuisante qu'une métamorphose réussie, mais qu'il soit \emph{maudit} s'il abandonnait maintenant.

Très bien, toutes ces conneries du dix-neuvième siècle pouvaient aller se faire voir.

La réalité n'était pas faite d'atomes, ce n'était pas un ensemble de petites boules de billard rebondissantes. C'était juste un autre mensonge. La notion que les atomes étaient des petits points était juste une autre hallucination commode à laquelle les gens s'accrochaient parce qu'ils ne voulaient pas se confronter à la forme inhumainement étrange de la réalité qui se cachait en dessous. Pas étonnant, donc, que ses tentatives de métamorphoser quelque chose en se basant sur cette illusion n'aient pas fonctionné. S'il voulait en être capable, il allait lui falloir abandonner son humanité et forcer ses pensées à se conformer aux véritables mathématiques de la mécanique quantique.

Il n'y avait \emph{pas de particules}, seulement des \emph{nuages d'amplitudes} dans des \emph{espaces de configurations multiparticulaires}, et ce que son cerveau s'imaginait tendrement être une gomme n'était rien d'autre qu'un immense \emph{facteur} dans une fonction d'onde que se \emph{trouvait être factorisable}, elle n'avait pas une existence distincte, pas plus qu'il n'existait un facteur solide et distinct nommé 3 caché dans le nombre 6, et si sa baguette était capable \emph{d'altérer les facteurs dans les fonctions d'onde approximativement factorisables}, alors elle devrait sacrément bien savoir altérer le facteur légèrement \emph{plus petit} que le cerveau de Harry visualisait sous la forme d'un petit bout de matière sur la gomme…

\later

Hermione fonçait le long des couloirs, ses chaussures battant la pierre avec force, sa respiration haletante, le choc de l'adrénaline courant à travers son corps.

Comme l'image d'une jeune femme devenant une vieillarde, comme un vase devenant deux visages.

Qu'avaient-ils fait~?

\emph{Qu'avaient-ils fait~?}

Elle parvint à la salle et ses doigts commencèrent par glisser sur la poignée, trop imprégnés de sueur, elle serra plus fort et la porte s'ouvrit…

… en un éclair de perception elle vit Harry qui regardait un petit rectangle rose sur la table en face de lui…

… tandis qu'à quelques pas de là le petit fil noir, presque invisible à cette distance, soutenait tout ce poids…

«~\emph{Harry, sors de la salle~!}~»

Une surprise absolue se dessina sur le visage de Harry, il se leva si vite qu'il en tomba presque, s'arrêtant uniquement pour saisir le petit rectangle rose, et il courut jusqu'à la porte, elle avait déjà fait un pas de côté, sa baguette était déjà dans sa main, montait déjà vers le fil…

«~\emph{Finite Incantatem~!}~»

Et Hermione referma violemment la porte juste quand depuis l'intérieur de la pièce retentit le bruit colossal de cent kilogrammes de métal s'effondrant sur le sol.

Elle haletait, cherchait de l'air, elle avait couru jusque-là sans s'arrêter, elle était trempée de sueur et ses jambes et ses cuisses brûlaient comme si elles étaient des flammes vivantes, elle n'aurait pas pu répondre aux questions de Harry même s'il lui avait offert tout l'or du monde.

Hermione cligna des yeux et se rendit compte qu'elle avait commencé à s'effondrer, que Harry l'avait rattrapée, et qu'il la faisait doucement descendre jusqu'au sol.

«~… sens bien… parvint-elle à murmurer.

--- \emph{Quoi~?}~» dit Harry, plus pâle qu'elle ne l'avait jamais vu.

«~… est-ce que… tu te sens… sens bien…~»

Harry commença à avoir l'air encore plus effrayé maintenant qu'il avait compris le sens de la question. «~Je, je ne pense pas avoir de symptôme…~»

Hermione ferma les yeux un instant. «~Bien, murmura-t-elle. Reprendre… mon souffle.~»

Cela prit un moment. Harry avait toujours l'air effrayé. Il valait mieux, peut-être que cela lui donnerait une leçon.

Hermione fouilla dans la bourse que Harry lui avait achetée, murmura «~eau~» à travers sa gorge asséchée, sortit la bouteille et but d'immenses gorgées.

Et il lui fallut un autre bon moment avant de pouvoir parler.

«~Nous avons enfreint les règles, Harry, dit-elle d'une voix cassée. Nous avons enfreint les règles.

--- Je… Harry déglutit. Je ne vois toujours pas en quoi, j'y ai \emph{réfléchi} mais…

--- J'ai demandé si la métamorphose était sûre et \emph{tu m'as répondu~!}~»

Il y eut une pause.

«~C'est tout~?~» dit Harry.

Elle aurait voulu hurler.

«~Harry, tu ne comprends pas~? dit-elle. C'est fait de petites fibres, et si elles se \emph{démêlaient}, qui \emph{sait} ce qui pourrait mal tourner, \emph{nous n'avons pas demandé au professeur McGonagall~!} Ne comprends-tu pas ce que nous étions en train de faire~? Nous faisions des expériences de métamorphose~! \emph{Nous faisions des expériences de métamorphose~!}~»

Il y eut une autre pause.

«~D'accord… dit lentement Harry. C'est probablement une de ces choses qu'ils ne prennent même pas la peine de nous dire de ne \emph{pas} faire parce que c'est trop évident. Ne pas tester de brillantes nouvelles idées de métamorphose seuls dans une salle vacante sans d'abord en parler à un professeur.

--- Tu aurais pu nous faire tuer, Harry~!~» Hermione savait que ce n'était pas juste, elle aussi avait commis l'erreur, mais elle était encore en colère contre lui, il avait toujours l'air tellement sûr de lui, et ça l'avait entraînée dans son sillage sans qu'elle y pense.

«~Nous aurions pu \emph{entacher le dossier parfait du professeur McGonagall} \emph{!}

--- Oui, dit Harry, ne lui parlons pas de ça, d'accord~?

--- Nous devons nous arrêter, dit Hermione. Sans quoi nous allons nous faire du mal. Nous sommes trop jeunes, Harry, nous ne pouvons pas faire ça, pas encore.~»

Un faible sourire passa sur le visage de Harry. «~Euh, là, tu as tort.~»

Et il brandit un petit rectangle rose, une gomme en caoutchouc avec un petit bout en métal.

Hermione la regarda, perplexe.

«~La mécanique quantique n'a pas suffi, dit Harry. J'ai dû descendre jusqu'à la physique intemporelle avant que ça fonctionne. J'ai dû voir la baguette comme quelque chose qui imposait une \emph{relation} entre deux réalités séparées, présente et future, au lieu de quelque chose qui \emph{changerait} quoi que ce soit au cours du temps -- mais je l'ai fait, Hermione, j'ai vu par-delà l'illusion des objets, et je parie que pas un seul autre sorcier au monde ne pourrait le faire. Même si un Moldu avait un jour entendu parler de la formulation intemporelle de la mécanique quantique, ça ne serait pour lui qu'une étrange croyance au sujet de trucs quantiques bizarres et distants, il n'aurait pas \emph{vu} que c'est la \emph{réalité}, il n'aurait pas accepté que le monde qu'il connaît n'est qu'une hallucination. J'ai métamorphosé une \emph{partie} de la gomme sans la changer \emph{en entier}.~»

Hermione leva de nouveau sa baguette et la pointa vers la gomme.

Pendant un instant, de la colère apparut sur le visage de Harry, mais il n'esquissa aucun geste pour l'arrêter.

«~\emph{Finite Incantatem}, dit Hermione. Vérifie auprès du professeur McGonagall avant de réessayer.~»

Harry hocha la tête, bien que son visage soit encore un peu pincé.

«~Et nous devons quand même arrêter, dit Hermione.

--- \emph{Pourquoi}~? dit Harry. Ne vois-tu pas ce que ça veut \emph{dire}, Hermione~? Les sorciers ne \emph{savent pas} tout~! Ils sont trop peu nombreux, et encore moins d'entre eux ont un savoir scientifique, ils n'ont pas épuisé toutes les solutions faciles…

--- Ce n'est pas \emph{sans danger}, dit Hermione. Si on \emph{peut} découvrir des choses nouvelles, c'est encore \emph{plus} dangereux~! On est \emph{trop jeunes}~! On a déjà fait une grosse erreur, on pourrait simplement \emph{mourir} la prochaine fois~!~»

Puis Hermione tressaillit.

Harry détourna son regard et commença à prendre des inspirations lentes et profondes.

«~N'essaie pas tout seul, Harry~», dit Hermione, la voix tremblante, «~s'il te plaît.~»

\emph{S'il te plaît, ne me fais pas décider si je dois aller voir le professeur Flitwick.}

Il y eut une longue pause.

«~Donc tu veux qu'on étudie~», dit Harry. Elle pouvait voir qu'il essayait de garder la colère hors de sa voix. «~Juste qu'on étudie.~»

Hermione ne savait pas si elle était censée répondre mais… «~Par exemple tu as étudié, euh, la physique intemporelle, c'est ça~?~»

Harry la regarda à nouveau.

«~Ce que tu as fait, dit timidement Hermione, ce n'était pas grâce à \emph{nos} expériences, n'est-ce pas~? Tu as pu le faire parce que tu as lu beaucoup de livres.~»

Harry ouvrit la bouche, puis il la referma. Il y avait un air frustré sur son visage.

«~Très bien, dit Harry. Que dis-tu de ça. On étudie, et si je pense qu'il y a quelque chose qui vaut \emph{vraiment} le coup d'être essayé, on essaiera après que j'ai demandé à un professeur.

--- D'accord~», dit Hermione. Elle ne s'effondra pas de soulagement, mais c'était seulement parce qu'elle était déjà assise.

«~On va déjeuner~?~» dit prudemment Harry.

Hermione hocha la tête. Oui, c'était une bonne idée. Un vrai déjeuner, cette fois.

Elle se releva précautionneusement, grimaçant quand son corps lui hurla dessus…

Harry pointa sa baguette vers elle et dit «~\emph{Wingardium Leviosa}.~»

Hermione battit des paupières lorsque l'énorme poids sur ses jambes diminua jusqu'à devenir supportable.

Un sourire jaillit sur le visage de Harry. «~On peut \emph{soulever} quelque chose sans le faire flotter complètement, dit-il. Tu te souviens de cette expérience~?~»

Hermione sourit en retour, incapable de s'en empêcher, même si elle pensait qu'elle aurait dû être encore en colère.

Et elle commença à marcher en direction de la Grande Salle, se sentant merveilleusement, remarquablement légère, tandis que Harry s'appliquait à garder sa baguette pointée sur elle.

Il ne parvint à maintenir le sort que cinq minutes, mais c'était l'intention qui comptait.

\later

Minerva regardait Dumbledore.

Dumbledore l'observait d'un air scrutateur. «~As-tu compris quoi que ce soit à ce qu'il vient de dire~?~» dit le directeur d'un air perplexe.

Ça avait été le charabia le plus incompréhensible qu'il ait jamais été donné à Minerva d'entendre. Elle se sentait un peu gênée d'avoir appelé le directeur pour qu'il vienne écouter ça, mais les ordres qu'elle avait reçus à ce sujet étaient formels.

«~J'ai bien peur que non, dit Minerva d'un ton guindé.

--- Donc~», dit Dumbledore. La barbe d'argent se balança loin d'elle et le regard pétillant du vieux sorcier changea à nouveau de cible. «~Tu penses être capable de faire quelque chose dont les autres sorciers ne sont pas capables, quelque chose que nous croyons être impossible.~»

Ils se tenaient tous les trois dans l'atelier de métamorphose privé du directeur, là où le brillant phénix du Patronus de Dumbledore avait dit à Minerva d'emmener Harry, quelques instants après que son propre Patronus eut averti Dumbledore. De la lumière tombait de la lucarne et illuminait le grand diagramme alchimique à sept branches dessiné au centre de la pièce circulaire, ce qui révélait son état légèrement poussiéreux et attristait Minerva. La recherche sur la Métamorphose était l'une des plus grandes joies de Dumbledore, et bien qu'elle ait su qu'il avait été assez pressé par le temps dernièrement, elle ne s'était pas rendu compte que c'était à ce point.

Et maintenant Harry Potter allait lui faire perdre encore plus de temps. Mais elle ne pouvait certainement pas en vouloir à \emph{Harry}. Il avait bien agi en venant la voir et en lui disant qu'il avait une idée liée à la Métamorphose qui était pour le moment considérée comme étant impossible, et elle avait elle-même fait ce qu'on lui avait dit de faire~: elle avait donné l'ordre à Harry de se tenir coi et de ne discuter de cela avec personne d'autre avant qu'elle ait consulté le directeur à ce sujet et qu'ils se soient rendu en un lieu sûr.

Si Harry avait commencé par \emph{décrire précisément} ce qu'il croyait pouvoir faire, elle n'aurait pas pris cette peine.

«~Écoutez, je sais que c'est difficile à expliquer, dit Harry d'un ton un peu gêné. La conséquence de tout cela, c'est que vous croyez quelque chose qui entre en conflit avec ce que les scientifiques croient, et ce dans un domaine où je m'attends vraiment à ce que les scientifiques en sachent plus que les sorciers.~»

Minerva aurait ostensiblement soupiré si Dumbledore n'avait eu l'air de prendre toute cette histoire avec autant de sérieux.

L'idée de Harry venait simplement de son ignorance, rien de plus. Si vous changiez la moitié d'une sphère de métal en verre, \emph{toute la sphère} acquérait une forme différente. Changer la partie, \emph{c'était} changer le tout, et cela impliquait d'enlever toute une forme et la remplacer par une autre. Qu'est-ce que cela pouvait même bien vouloir \emph{dire} de métamorphoser seulement la moitié d'une sphère de métal~? Que la sphère \emph{dans son ensemble} avait la même forme qu'avant, mais que \emph{la moitié} de la sphère avait une forme différente~?

«~M. Potter, dit le professeur McGonagall, ce que vous voulez faire n'est pas seulement impossible, c'est \emph{illogique}. Si vous avez changé la moitié de quelque chose, vous en \emph{avez} changé le tout.

--- En effet, dit Dumbledore. Mais Harry est le héros, il pourrait donc être capable de faire des choses logiquement impossibles.~»

Minerva aurait levé les yeux au ciel si elle n'avait pas été depuis bien longtemps insensible à ce genre de remarques.

«~En admettant que \emph{ce soit} possible, dit Dumbledore, peux-tu imaginer la moindre raison pour laquelle les résultats seraient d'une quelconque façon différents de ceux d'une métamorphose ordinaire~?~»

Minerva fronça les sourcils. Le fait que le concept soit littéralement inimaginable lui donnait quelque difficulté, mais elle essaya de prendre l'idée au pied de la lettre. Une métamorphose appliquée à uniquement la moitié d'une sphère de métal…

«~Des choses étranges se passeraient à la jointure~? dit Minerva. Mais ça ne serait pas différent de métamorphoser un objet entier en une forme composée de deux parties…~»

Dumbledore hocha la tête.

«~C'est ce que je pensais moi aussi. Et Harry, si ta théorie est correcte, cela veut dire que ce que tu comptes faire est \emph{exactement} comme une métamorphose, mais appliqué à la partie d'un sujet plutôt qu'au sujet dans son ensemble~? \emph{Aucun} autre changement~?

--- Oui, dit Harry d'un ton arrêté. C'est tout l'intérêt.~»

Dumbledore la regarda de nouveau.

«~Minerva, peux-tu imaginer la moindre raison pour laquelle ce serait dangereux~?

--- Non, dit Minerva après qu'elle eut fini de sonder sa mémoire.

--- Moi non plus, dit le directeur. Très bien, puisque ce devrait être en tout point analogue à une métamorphose ordinaire, et puisque nous ne pouvons imaginer aucune raison pour laquelle ce serait dangereux, je pense qu'une prudence de niveau deux devrait suffire.~»

Minerva fut surprise, mais elle ne fit pas d'objection. Dumbledore était de loin son aîné en matière de métamorphose, et il avait littéralement essayé des milliers de nouvelles métamorphoses sans jamais choisir un niveau de prudence trop bas. Il avait utilisé la métamorphose \emph{en combat} et il était \emph{toujours en vie}. Si le directeur pensait que le deuxième niveau suffisait, ça suffisait.

Le fait que Harry allait certainement échouer était bien sûr sans rapport avec ce sujet.

Les deux sorciers commencèrent à mettre en place les lieux et à installer des toiles de détection. La toile la plus importante était celle qui s'assurait qu'aucun matériau transfiguré n'entre dans l'air. Harry serait enfermé dans un champ de force séparé alimenté par sa propre source d'oxygène, juste pour être certain, et seule sa baguette serait autorisée à sortir du bouclier, et la jointure entre les deux serait extrêmement serrée. Ils étaient à Poudlard et ne pouvaient donc pas transplaner hors du domaine tout matériau qui montrerait des signes de combustion spontanée, mais ils pouvaient le jeter par la lucarne presque aussi rapidement, et c'est pour cette raison que toutes les fenêtres s'ouvraient vers l'extérieur. Harry lui-même serait éjecté par une autre lucarne au premier contretemps.

Harry les regarda travailler avec un air un peu effrayé.

«~Ne vous en faites pas M. Potter, dit le professeur McGonagall au milieu de sa description de l'installation, ce sera presque certainement inutile. Si nous nous \emph{attendions} à ce que quoi que ce soit tourne mal, vous ne seriez pas autorisé à essayer. Ce sont juste des précautions ordinaires pour toute métamorphose que personne n'a essayé auparavant.~»

Harry déglutit et acquiesça.

Quelques minutes plus tard, Harry était attaché à une chaise de sûreté, baguette posée contre une sphère de métal -- une sphère qu'il aurait due, selon les résultats de ses derniers contrôles, mettre au moins une demi-heure à métamorphoser.

Et quelques minutes après \emph{ça}, Minerva était appuyée contre un mur, se sentant au bord de l'évanouissement.

Il y avait un petit morceau de verre sur la sphère, là où la baguette de Harry était appuyée.

Harry évita un \emph{je vous l'avais bien dit}, mais l'air suffisant sur son visage le disait à sa place.

Dumbledore jetait des sorts d'analyse sur la sphère, une expression de plus en plus intriguée sur le visage. Trente ans venaient de s'envoler de ses traits.

«~Fascinant, dit Dumbledore. C'est exactement ce qu'il avait annoncé. Il a simplement métamorphosé une partie du sujet sans en métamorphoser le tout. Harry, tu dis que ce n'est en réalité qu'une limite conceptuelle~?

--- Oui, dit Harry, mais une limite profonde, le seul fait de savoir que ça devait être une limite conceptuelle n'a pas suffi. J'ai dû refouler la partie de mon esprit qui commettait l'erreur et la remplacer par une pensée au sujet de la véritable réalité sous-jacente aux choses, celle que les scientifiques ont découverte.

--- Vraiment fascinant, dit Dumbledore. Je suppose que pour qu'un autre sorcier reproduise ce que tu as fait, il lui faudrait des mois d'études, et encore~? Et pourrais-je te demander de métamorphoser partiellement quelques autres sujets~?

--- Probablement que oui et bien sûr~», répondit Harry.

Une demi-heure plus tard, Minerva se sentait toujours autant abasourdie, mais considérablement rassurée quant aux questions de sûreté.

C'\emph{était} bien la même chose, mis à part le fait que ce soit logiquement impossible.

«~Je pense que cela suffira, directeur, dit enfin Minerva. Je soupçonne que la métamorphose partielle est plus fatigante que la variante ordinaire.

--- De moins en moins avec la pratique, dit le garçon pâle et épuisé d'une voix instable, mais ouais, vous avez raison à ce sujet.~»

Le processus consistant à extraire Harry du dispositif prit une minute de plus, et Minerva l'escorta ensuite jusqu'à une chaise plus confortable, suite à quoi Dumbledore fit surgit un soda à la crème glacée.

«~\emph{Félicitations}, M. Potter~!~» dit le professeur McGonagall, et elle était sincère. Elle aurait été prête à parier presque n'importe quoi que ça n'allait pas fonctionner.

«~Félicitations, en effet, dit Dumbledore. Même moi, je n'ai fait aucune découverte originale en Métamorphose avant l'âge de quatorze ans. Depuis l'époque de Dorotea Senjak, Jamais génie n'avait si rapidement éclos.

--- Merci, dit Harry, l'air un peu surpris.

--- Néanmoins, dit Dumbledore d'un ton pensif, je pense qu'il serait fort sage de garder cet heureux événement secret, du moins pour l'instant. Harry, as-tu parlé de ton idée avec qui que ce soit d'autre avant d'en parler au professeur McGonagall~?~»

Il y eut un silence.

«~Euh… dit Harry. Je ne veux vendre personne à l'Inquisition, mais il y a bien quelqu'un d'autre à qui…~»

Les mots explosèrent presque à travers les lèvres du professeur McGonagall~:

«~\emph{Quoi}~? Vous avez discuté d'une forme de métamorphose totalement nouvelle avec un \emph{élève} avant de consulter une autorité reconnue~? Avez-vous la moindre idée d'à quel point c'était \emph{irresponsable}~?

--- Je suis désolé, dit Harry. Je ne m'en suis pas rendu compte.~»

Le garçon avait l'air correctement apeuré, et Minerva sentit quelque chose se détendre à l'intérieur d'elle-même. Au moins Harry comprenait à quel point il avait été insensé.

«~Tu dois faire jurer le secret à mademoiselle Granger, dit Dumbledore avec le plus grand sérieux. Et n'en parle à personne d'autre à moins que tu n'aies une raison particulièrement bonne de le faire, et cette personne devra jurer elle aussi.

--- Ah… pourquoi~? dit Harry.

--- Parce que tu peux faire quelque chose que personne ne te croira capable de faire, dit Dumbledore. Quelque chose de totalement inattendu. Cela pourrait se révéler ton avantage capital, Harry, et nous devons le préserver. S'il te plaît, fais-moi confiance à ce sujet.~»

Le professeur McGonagall hocha la tête, son visage ferme ne montrant rien de sa confusion intérieure.

«~S'il vous plaît M. Potter, faites-le, dit-elle.

--- Très bien… dit lentement Harry.

--- Lorsque nous aurons fini d'examiner vos sujets de métamorphose, ajouta Dumbledore, vous pourrez pratiquer la métamorphose partielle, verre vers acier et acier vers verre \emph{uniquement}, avec mademoiselle Granger comme guetteuse. Naturellement, si l'un de vous suspecte le moindre symptôme d'une quelconque forme de maladie de métamorphose, informez-en un professeur immédiatement.~»

Juste avant qu'il ne quitte la pièce, la main sur la poignée de la porte, Harry se retourna et dit~:

«~Pendant qu'on est tous là, l'un de vous a-t-il remarqué quoi que ce soit de différent au sujet du professeur Rogue~?

--- Différent~?~» dit le directeur.

Minerva ne laissa pas son sourire narquois apparaître sur son visage. Bien sûr que le garçon éprouvait une appréhension vis-à-vis du “maléfique maître de Potions”, puisqu'il n'avait aucun moyen de savoir la raison pour laquelle on pouvait faire confiance à Severus. Ç'aurait été pour le moins étrange d'expliquer à Harry que Severus était toujours amoureux de sa mère.

«~Je veux dire, son comportement a-t-il récemment changé de quelque façon que ce soit~? dit Harry.

--- Pas que j'ai pu constater… dit lentement le directeur. Pourquoi poses-tu cette question~?~»

Harry secoua sa tête. «~Je ne veux pas créer de préconception dans vos observations en le disant. Peut-être que vous pourriez juste ouvrir l'œil~?~»

Cela envoya un frémissement de malaise à travers le corps de Minerva d'une façon qu'aucune accusation directe à l'encontre de Severus n'aurait pu accomplir.

Harry s'inclina respectueusement devant eux et prit congé.

\later

«~Albus, dit Minerva après que le garçon fut parti, comment \emph{saviez-vous} qu'il fallait prendre Harry au sérieux~? J'aurais considéré son idée comme étant simplement impossible~!~»

Le vieux sorcier affichait un visage très sérieux. «~Pour la même raison qu'elle doit être gardée secrète, Minerva. Pour la même raison que je t'ai dit de venir me voir si Harry déclarait quoi que ce soit de cette sorte. Parce que c'est un pouvoir que Voldemort ignore.~»

Elle mit quelques secondes à absorber le sens des mots.

Et le frisson glacé descendit le long de son épine dorsale, comme à chaque fois qu'elle se souvenait.

Ça avait commencé comme un entretien d'embauche ordinaire, Sybill Trelawney postulant pour le poste de professeur de Divination.

\prophesy{Celui qui a le pouvoir de vaincre le seigneur des ténèbres approche,\\ il naîtra de ceux qui l'ont par trois fois défié,\\ il sera né lorsque mourra le septième mois,\\ et le Seigneur des ténèbres le marquera comme son égal,\\ mais il aura un pouvoir que le Seigneur des ténèbres ignore,\\ et l'un devra détruire l'autre, n'en laissant qu'un vestige,\\ car ces deux différents esprits ne peuvent exister dans le même monde.}

Ces mots effroyables, prononcés par cette terrible voix tonitruante, ne semblait pas correspondre à quelque chose comme la métamorphose partielle.

«~Peut-être que non alors, dit Dumbledore après que Minerva eut essayé de le lui expliquer. J'avoue avoir espéré quelque chose qui nous aide à trouver l'horcruxe de Voldemort, où qu'il puisse être caché. Mais…~» Le vieux sorcier haussa les épaules. «~Les prophéties sont des choses épineuses Minerva, et il vaut mieux ne prendre aucun risque. Le plus petit détail pourrait se révéler décisif s'il peut rester inattendu.

--- Et que pensez-vous qu'il ait voulu dire au sujet de \emph{Severus}~? dit Minerva.

--- Là, je n'en ai aucune idée, soupira Dumbledore. À moins que Harry ne soit en train de jouer un coup contre Severus et qu'il ait pensé qu'une question ouverte pourrait être prise sérieusement là où une allégation directe aurait été rejetée. Et si c'est en effet ce qui s'est passé, Harry a eu raison en pensant que je ne penserais pas que ce serait le cas. Gardons simplement l'œil ouvert, sans préconceptions, comme il l'a demandé.~»

\latersection{Après-coup, 1~:}

«~Euh, Hermione~? dit Harry d'une toute petite voix. Je crois que je te dois de très, très grandes excuses.~»

\latersection{Après-coup, 2~:}

Alissa Cornfoot regardait le maître de Potions, les yeux légèrement vitreux, tandis qu'il donnait un cours sévère à sa classe, tenant un petit haricot de bronze à la main et disant quelque chose au sujet de flaques de chair humaine hurlantes. Depuis le début de l'année, elle avait eu du mal à écouter en cours de Potions. Elle n'arrêtait pas de regarder leur horrible professeur méchant et graisseux et à fantasmer au sujet de retenues spéciales. Il y avait probablement quelque chose qui n'allait \emph{pas du tout} chez elle, mais elle ne pouvait pas s'empêcher de continuer…

«~Ouille~!~» dit alors Alissa.

Rogue venait d'envoyer le haricot de bronze sur le front d'Alissa avec une infaillible précision.

«~Mademoiselle Cornfoot~», dit le maître de Potions, la voix coupante, «~c'est une potion délicate à réaliser, et si vous n'arrivez pas à faire attention, vous ferez du mal à vos camarades, pas seulement à vous-même. Venez me voir après le cours.~»

Les quatre derniers mots ne l'aidèrent absolument pas, mais elle essaya de plus belle et parvint au terme de la journée sans faire fondre personne.

Après le cours, Alissa s'approcha du bureau. Une partie d'elle-même voulait se tenir là timidement, le visage confus, les mains jointes dans son dos avec repentance, juste au cas où, mais un sourd instinct lui dit que ce pourrait être une \emph{mauvaise idée}. Alors elle s'avança, le visage neutre, avec une posture tout à fait appropriée pour une jeune demoiselle, et dit~:

«~Professeur~?

--- Mademoiselle Cornfoot, dit Rogue sans lever les yeux des contrôles qu'il corrigeait, je ne vous renvoie pas vos affections, je commence à trouver vos regards dérangeants, et vous retiendrez vos yeux à partir d'aujourd'hui. Est-ce assez clair~?

--- Oui~», dit Alissa dans un glapissement étranglé, et Rogue la congédia, et elle fuit la salle, ses joues en feu comme si elles avaient été faites de lave en fusion.

%  LocalWords:  kay Incantatemed Dorotea Senjak
